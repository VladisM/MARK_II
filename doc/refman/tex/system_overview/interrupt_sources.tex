\subsection{Interrupt sources}

In the table \ref{tab:intsources} are listed all interrupt vectors. Destinations
addresses of interrupts are hardwired into CPU and cannot be changed. All
interrupts are mascable using interrupt controller.

Support for nested interrupts isn't implemented. When one interrupt come, CPU
will jump into interrupt service routine and until RETI instruction is executed,
another interrupt cannot be raised. Anyway, when an interrupt is active, another
one is then put into queue.

\begin{table}[h]
    \centering
    \begin{tabular}{|l|l|l|}
        \hline
        \textbf{Int number} & \textbf{Peripheral} & \textbf{Purpose}                \\ \hline
        0                   & SWI                 & Software interrupt              \\ \hline
        1                   & System Timer        & SysTim compare match / overflow \\ \hline
        2                   & -                   & -                               \\ \hline
        3                   & -                   & -                               \\ \hline
        4                   & -                   & -                               \\ \hline
        5                   & -                   & -                               \\ \hline
        6                   & -                   & -                               \\ \hline
        7                   & -                   & -                               \\ \hline
        8                   & UART0               & Serial port 0                   \\ \hline
        9                   & UART1               & Serial port 1                   \\ \hline
        10                  & UART2               & Serial port 2                   \\ \hline
        11                  & PS2 0               & Byte received PS2 0             \\ \hline
        12                  & PS2 1               & Byte received PS2 1             \\ \hline
    \end{tabular}
    \caption{List of all interrupt sources}
    \label{tab:intsources}
\end{table}
