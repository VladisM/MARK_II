\subsection{Interrupt sources}

In the table \ref{tab:intsources} are listed all interrupt vectors. Destinations
addresses of interrupts are hardwired into CPU and cannot be changed. All
interrupts are mascable using interrupt controller.

Support for nested interrupts isn't implemented. When one interrupt come, CPU
will jump into interrupt service routine and until RETI instruction is executed,
another interrupt cannot be raised. Anyway, when an interrupt is active, another
one is then put into queue.

\begin{table}[h]
    \centering
    \begin{tabular}{|l|l|l|l|}
        \hline
        \textbf{Int number} & \textbf{Peripheral} & \textbf{Vector} & \textbf{Purpose}               \\ \hline
        0                   & System Timer        & 0x000010        & Timer compare match / overflow \\ \hline
        1                   & -                   & 0x000012        & -                              \\ \hline
        2                   & -                   & 0x000014        & -                              \\ \hline
        3                   & -                   & 0x000016        & -                              \\ \hline
        4                   & -                   & 0x000018        & -                              \\ \hline
        5                   & -                   & 0x00001A        & -                              \\ \hline
        6                   & -                   & 0x00001C        & -                              \\ \hline
        7                   & -                   & 0x00001E        & -                              \\ \hline
        8                   & UART0 Tx            & 0x000020        & Byte sent                      \\ \hline
        9                   & UART0 Rx            & 0x000022        & Byte received                  \\ \hline
        10                  & UART1 Tx            & 0x000024        & Byte sent                      \\ \hline
        11                  & UART1 Rx            & 0x000026        & Byte received                  \\ \hline
        12                  & UART2 Tx            & 0x000028        & Byte send                      \\ \hline
        13                  & UART2 Rx            & 0x00002A        & Byte received                  \\ \hline
        14                  & Timer 0             & 0x00002C        & Timer compare match / overflow \\ \hline
        15                  & Timer 1             & 0x00002E        & Timer compare match / overflow \\ \hline
        16                  & Timer 2             & 0x000030        & Timer compare match / overflow \\ \hline
        17                  & Timer 3             & 0x000032        & Timer compare match / overflow \\ \hline
        18                  & PS2 Keyboard        & 0x000034        & Byte received                  \\ \hline
        19                  & -                   & 0x000036        & -                              \\ \hline
        20                  & -                   & 0x000038        & -                              \\ \hline
        21                  & -                   & 0x00003A        & -                              \\ \hline
        22                  & -                   & 0x00003C        & -                              \\ \hline
        23                  & -                   & 0x00003E        & -                              \\ \hline
        24                  & -                   & 0x000040        & -                              \\ \hline
        25                  & -                   & 0x000042        & -                              \\ \hline
        26                  & -                   & 0x000044        & -                              \\ \hline
        27                  & -                   & 0x000046        & -                              \\ \hline
        28                  & -                   & 0x000048        & -                              \\ \hline
        29                  & -                   & 0x00004A        & -                              \\ \hline
        30                  & -                   & 0x00004C        & -                              \\ \hline
        31                  & -                   & 0x00004E        & -                              \\ \hline
    \end{tabular}
    \caption{List of all interrupt sources}
    \label{tab:intsources}
\end{table}


