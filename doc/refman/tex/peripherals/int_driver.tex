\subsection{Interrupt driver}

Interrupt driver is one of peripherals tightly connected to the CPU. It is
responsible for taking care about interrupts. It can disable some interrupt
vectors, it also have to decide what interrupt should be evaluated when more
request come at once. Also, interrupt driver have to take care about interrupt
queue where interrupt requests are waiting for CPU time.

\subsubsection{Function}

There is only one register in interrupt driver, from programmer look this
peripheral is quiet simple. This one register is INTMR and is 32bit wide. There
is also 32bit sources of interrupts. Interrupt driver is responsible for
masking disabled interrupt requests. You can disable interrupt by writing
logical one into register INTMR. Bit no. 0 is controlling interrupt source 0,
bit 1 interrupt source 1 and so on.

Priority of individuals interrupts are decreasing with interrupt number, for
example interrupt 0 is smallest number, so it have highest priority, interrupt
31 is largest interrupt number, so have lowest priority.

\subsubsection{Registers}

There is only one register called INTMR, his offset is $+0$ and thus it can be
found at base address.

\begin{table}[h]
    \centering
    \begin{tabular}{|l|l|l|}
        \hline
        \textbf{Offset} & \textbf{Name} & \textbf{Purpose}             \\ \hline
        $+0$            & INTMR         & Mask for enabling interrupts \\ \hline
    \end{tabular}
    \caption{Interrupt driver registers map}
    \label{tab:int_driver_reg_map}
\end{table}

\subsubsection{Hacking}

Interrupt driver can't be modified because it is highly coupled with CPU. When
you want to modify SoC, you have every time had interrupt controlled in your design.

There is of course standard bus interface for this peripheral, and some device
specific signals. These are int\_req, int\_accept, int\_completed and int\_cpu\_req.

Signals int\_accept, int\_completed and int\_cpu\_req should be connected directly
to the CPU and are used for communication between CPU and interrupt driver.

Signal int\_req is 32 bits wide vector of interrupt request signals. These should be
connected to the peripherals interrupt request outputs.

\begin{lstlisting}[language=VHDL, frame=single]
entity intController is
    generic(
        BASE_ADDRESS: unsigned(23 downto 0) := x"00000"
    );
    port(
        clk: in std_logic;
        res: in std_logic;
        address: in unsigned(23 downto 0);
        data_mosi: in unsigned(31 downto 0);
        data_miso: out unsigned(31 downto 0);
        WR: in std_logic;
        RD: in std_logic;
        ack: out std_logic;
        int_req: in std_logic_vector(31 downto 0);
        int_accept: in std_logic;
        int_completed: in std_logic;
        int_cpu_req: out std_logic_vector(31 downto 0)
    );
end entity intController;
\end{lstlisting}

