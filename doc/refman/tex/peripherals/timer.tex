\subsection{Timer}

This is relative complex peripheral of MARK-II. Timer is 16bit timer, witch can
generate interrupt on compare match or overflow, and also can generate two PWM
outputs with 16b resolution.

\subsubsection{Function}

Timer have prescaler with can divide system clock by 1, 2, 4 or 8. Core counter
is then clocked at $F_{clk}/prescaler$, timer is start counting from zero to
up, until reach end condition.

End condition can be simply timer overflow or compare match with value in
register OCRA. When clear on compare match is enabled, then when $Timer value
== OCRA$ clear signal is generated and timer is reseted. If compare interrupt
is enabled, Timer will generate interrupt too. Interrupt can be generated also
when counter overflow.

Timer can also generate PWM outputs, in that situation, you should disable
clear on compare because OCRA value is used to control duty cycle. OCRB is
used to control duty cycle of second channel of PWM output.

PWM generation work in following way:

\begin{itemize}
    \item Timer is equal to 0, counting started, PWM output is in logical one.
    \item Counting up.
    \item Compare match occur, PWM output is set to logical zero, counting continues.
    \item Still counting up.
    \item Timer is equal to top, PWM output is set to logical zero, counter is overflow to zero and counting continues with next period.
\end{itemize}

Anytime when write access to timer value register called TCNR is performed,
timer is reseted to zero in same way as with System timer.

\subsubsection{Registers}

All registers are listed in table \ref{tab:tim_reg_map}. Registers OCRA and OCRB
are both 16 bits wide and are used for generating PWM, OCRA is also used as top
if timer is configured to clear on compare match.

Register TCNR is 16bit wide and hold actual value of counter. It can be read to
obtain this value, or it can be written, but when write access occur, counter is
reseted.

Register TCCR is control register. It is only 7 bits wide. Meaning of individual
bits are in table \ref{tab:tim_tccr_bits}.

\begin{table}[h]
    \centering
    \begin{tabular}{|l|l|l|}
        \hline
        \textbf{Offset} & \textbf{Name} & \textbf{Purpose}           \\ \hline
        $+0$            & TCCR          & Timer control register.    \\ \hline
        $+1$            & OCRA          & Output compare register A. \\ \hline
        $+2$            & OCRB          & Output compare register B. \\ \hline
        $+3$            & TCNR          & Actual value of timer.     \\ \hline
    \end{tabular}
    \caption{Timer register map}
    \label{tab:tim_reg_map}
\end{table}

\begin{table}[h]
    \centering
    \begin{tabular}{|l|l|}
        \hline
        \textbf{Bit} & \textbf{Purpose}                                                         \\ \hline
        $0$          & Enable clear on compare match.                                           \\ \hline
        $1$          & Enable timer.                                                            \\ \hline
        $2$          & Enable interrupt on overflow.                                            \\ \hline
        $3$          & Enable interrupt on compare match.                                       \\ \hline
        $4$          & Enable both channels of PWM.                                             \\ \hline
        $5$          & clksel0 - Select prescaler, see table \ref{tab:tim_prescaler_setting}.   \\ \hline
        $6$          & clksel1 - Select prescaler, see table \ref{tab:tim_prescaler_setting}.   \\ \hline
    \end{tabular}
    \caption{TCCR bits.}
    \label{tab:tim_tccr_bits}
\end{table}

\begin{table}[h]
    \centering
    \begin{tabular}{|l|l|l|}
        \hline
        \textbf{clkse1} & \textbf{clksel0} & \textbf{Prescaler} \\ \hline
        $0$ & $0$ & $F_{clk}/1$ \\ \hline
        $0$ & $1$ & $F_{clk}/2$ \\ \hline
        $1$ & $0$ & $F_{clk}/4$ \\ \hline
        $1$ & $1$ & $F_{clk}/8$ \\ \hline
    \end{tabular}
    \caption{Timer prescaler settings.}
    \label{tab:tim_prescaler_setting}
\end{table}

\subsubsection{Hacking}

There is nothing to change with arguments because timer is complex peripheral,
but you may want add one or more instance of timer.

\begin{lstlisting}[language=VHDL, frame=single]
entity timer is
    generic(
        BASE_ADDRESS: unsigned(23 downto 0) := x"000000"
    );
    port(
        clk: in std_logic;
        res: in std_logic;
        address: in unsigned(23 downto 0);
        data_mosi: in unsigned(31 downto 0);
        data_miso: out unsigned(31 downto 0);
        WR: in std_logic;
        RD: in std_logic;
        ack: out std_logic;
        enclk2: in std_logic;
        enclk4: in std_logic;
        enclk8: in std_logic;
        pwma: out std_logic;
        pwmb: out std_logic;
        intrq: out std_logic
    );
end entity;
\end{lstlisting}

Except classic bus interface, timer have some peripheral specific signals.
These are signals enclk2, enclk4, enclk8, pwma, pwmb and intrq. Signal intrq
is used for connecting into interrupt driver. Signals pwma and pwmb are outputs
from PWM generators A and B.

Signals enclk2, enclk4 and enclk8 are inputs, and they should be connected to
clock driver. They are control signals for prescaler.
