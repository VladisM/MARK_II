\subsection{PS2 driver}

PS2 driver is simple peripherals that allow connect keyboard or mouse. Driver
in this revision is able only to receive.

\subsubsection{Function}

This peripheral doesn't need any configuration. Just enable interrupt in interrupt
driver and is done. When key on connected keyboard are pressed, key code is
sent to the PS2 driver. When sending is completed, interrupt is called.

\subsubsection{Registers}

There is only one register, called PSBR, and it have offset $+0$. After key code
is received, it is stored in this register. Register is 8 bits wide.

\begin{table}[h]
    \centering
    \begin{tabular}{|l|l|l|}
        \hline
        \textbf{Offset} & \textbf{Name} & \textbf{Purpose}                 \\ \hline
        $+0$            & PSBR          & Last received byte from PS2 bus. \\ \hline
    \end{tabular}
    \caption{PS2 driver register map}
    \label{tab:ps2_reg_map}
\end{table}

\subsubsection{Hacking}

There is nothing to change on this peripheral. Interface is also standard.
There is only three device specific signals, ps2clk, ps2dat and intrq. Signal
intrq should be connected to the interrupt driver and signal ps2clk and ps2dat
should be connected to the top level pins.

\begin{lstlisting}[language=VHDL, frame=single]
entity ps2 is
    generic(
        BASE_ADDRESS: unsigned(23 downto 0) := x"000000"
    );
    port(
        clk: in std_logic;
        res: in std_logic;
        address: in unsigned(23 downto 0);
        data_miso: out unsigned(31 downto 0);
        RD: in std_logic;
        ack: out std_logic;
        ps2clk: in std_logic;
        ps2dat: in std_logic;
        intrq: out std_logic
    );
end entity ps2;
\end{lstlisting}
